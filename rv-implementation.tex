\subsection{Sidebar: Random Variable Implementation}

While the operations in the examples may seem magic, we wanted to give a peek
into the implementation to show they are in fact quite straight forward. SymPy
uses fairly general expression trees that are operated on by the various
functions to produce new trees.  To add a domain specific symbol, one needs to
do only a few things:

\begin{enumerate}
\item Create a class inheriting from the {\tt Symbol} class. 
\item Put all canonicalization of a symbol into the {\tt \_\_new\_\_} method, this
  insures that symbols can take advantage of SymPy's caching system.  Arguments
  are stored in an {\tt args} array for functions to inspect.
\item Add domain specific functions to a distinct module which manipulate SymPy expressions
\item Add special methods for printing, integration, and other common
  operations.  These usually start with {\tt \_eval\_}, and if not provided will
  be substituted with default representations.
\end{enumerate}

For random variables, we first create the abstract {\tt PSpace} class to hold
information about underlying probability spaces.  The {\tt RandomSymbol}
class inherits from the {\tt Symbol} class and represents the implemented {\tt
  PSpace} within SymPy expressions. 
The statistical operators \code{P, E, Density, ...} inspect the expression tree, extract the active probability spaces represented within, and construct a new non-statistical expression to compute the desired quantity. 

In the case of continuous random variables the statistical operators turn a statistical expression (i.e. \code{P(T>33)}) into a non-statistical integral expression (i.e. $\int_{33}^{\infty} \frac{\sqrt{2} e^{- \frac{1}{18} \left(x -30\right)^{2}}}{6 \sqrt{\pi}}\, dx $). For discrete random expressions (i.e. dice) form iterators or summation expressions. Multivariate random variables form matrix expressions, etc....
In each case SymPy-Stats offloads all computational work onto other more specialized and powerful modules. This allows SymPy-Stats to leverage the power and continuous growth of the more mature core modules.

While the project required several improvements to SymPy core, it does show the
power of the simple language approach SymPy uses.  We include these
implementation details in the hopes that the reader will be able to add her own
domain specific symbols.
