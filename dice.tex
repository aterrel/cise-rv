\section{Example 1 - Dice}

Random variables are formally functions on probability spaces.  Practically, they
represent ranges of values each with a certain probability.  SymPy is a
symbolic algebra system embedded in the Python programming language. In
addition to numeric variables like $x = 3$ it can also manipulate symbolic
variables $ x = $Symbol$('x')$. SymPy now includes a random variable type which
we can use to represent uncertainity in variables.  For example take a simple
six--sided die.

\begin{lstlisting}
>>> from sympy.stats import *
>>> X = Die(6) # a six sided die
\end{lstlisting}

X represents the value of a single die roll. It can take on the values
$1,2,3,4,5$ or $6$, each with probability $\frac{1}{6}$. In general these
variables act like normal symbolic variables and can participate in normal
mathematical expressions. For example, if you want to modify a dice roll by a
handicap, you can just add an integer to it.

\begin{lstlisting}
>>> mod_X = X + 3
\end{lstlisting}

While we use the {\tt Die} function above as a syntactic sugar, one can
represent any finite probability space by providing the density function, a map
from value to probability summing to one, and a symbol name, a default dummy is
provided.  For example, we could have written:
\begin{lstlisting}
>>> from sympy import Rational
>>> prob = Rational(1, 6)
>>> density_map = {1:prob, 2:prob, 3:prob, 4:prob, 5:prob, 6:prob}
>>> X = FiniteRV(density_map, 'die0' )
\end{lstlisting}

The difference between a random variables and a normal symbol can be seen when
using statistical operators, see Table~\ref{tab:stat_ops}. Here SymPy turns
expressions, e.g. {\tt 2*X + 4}, into some computed result. Additionally,
sometimes you need to augment the underlying probability space based on
knowledge of the space or with respect to an event, see Table~\ref{tab:cond_ops}.

\begin{table}[h]
\begin{tabular}{|lll|}
\hline
Operators & SymPy function & example\\ \hline
probability & {\tt P} & {\tt P(X < 3)}  -$> 1/3$\\ \hline
expectation & {\tt E} & {\tt E(X + 3)}  -$> 13/2$ \\ \hline
variance & {\tt var} & {\tt var(X + 3)} -$> 35/12$\\ \hline
density & {\tt var} & {\tt Density(X + 3)}  -$> \{4: 1/6, 5: 1/6, 6: 1/6,$ \\
            & & \phantom{Density(X+3)  -$> \{$ } $\quad 7: 1/6,  8: 1/6, 9: 1/6\}$ \\ \hline
\end{tabular}
\label{tab:stat_ops}
\caption{Operators over expressions using random variables}
\end{table}

\begin{table}[h]
\begin{tabular}{|lll|}
\hline
Condition & SymPy function & example\\ \hline
give an expression a value & {\tt Given} & {\tt Given(X < 3)} \\ \hline
set the value of an expression & {\tt Where} & {\tt Where(X < 3)} \\ \hline
\end{tabular}
\label{tab:cond_ops}
\caption{Conditions to change the underlying probability space}
\end{table}

Now that we have a few basic operations on expression we can ask a few more
interesting questions. For example, what is the probabilities related to two
dice rolls:
\begin{lstlisting}
>>> X = Die(6)
>>> Y = Die(6)

>>> Density(X) # PDF of a six sided die
{1: 1/6, 2: 1/6, 3: 1/6, 4: 1/6, 5: 1/6, 6: 1/6}
>>> Density(cos(pi*X)) # complex statements are fine too
{-1: 1/2, 1: 1/2}
>>> E(X+Y) # expected value of the sum of two dice
7
>>> P(Y>4)
1/3
>>> P(X+Y>=10, X<=4) # conditional probability X+Y>=10 given that X<=4
1/6
>>> Where(2*X < 8) # the set of values that twice a dice roll is less than
8
Domain: Or(die4 == 1, die4 == 2, die4 == 3)
\end{lstlisting}

By introducing a random variable type into the SymPy algebraic system, we have
created a language for modeling and querying symbolic statistical systems.
While these simple examples using dice are minimal, they are the basis for more
sophisticated examples.  Even with these simple operations, modeling discrete
systems become a simple readable process.
