\section{Example 1 - Dice}
%SymPy is a symbolic algebra system embedded in the Python programming language.

Formally random variables are functions on probability spaces.  Practically,
they represent ranges of values, each with a certain probability.  In addition
to numeric variables like $x = 3$, SymPy can also manipulate symbolic
variables, $x = \text{Symbol}('x')$. SymPy now includes a random variable type
which is used to represent uncertainty in variables.  For example, one can
declare and ask questions about a set of simple six--sided dice.

\begin{lstlisting}
>>> from sympy.stats import *
>>> X = Die(6) # a six sided die
>>> Y = Die(6) # a six sided die
>>> P(X>3) # probability that X is greater than 3
1/2
>>> E(X+Y) # expectation of the sum of two dice
7
\end{lstlisting}

X represents the value of a single die roll. It can take on the values
$1,2,3,4,5$ or $6$, each with probability $\frac{1}{6}$. In general these
variables act like normal symbolic variables and can participate in normal
mathematical expressions. For example, if you want to modify a die roll by a
handicap, you can add an integer to it.

\begin{lstlisting}
>>> mod_X = X + 3
\end{lstlisting}

The difference between a random variable and a normal symbol can be seen when
using the statistical operators described in Table~\ref{tab:stat_ops}.  Here
SymPy turns statistical expressions like {\tt E(2*X + 4)} into a computed
result. Additionally, one can compute statistics of the expression conditioned
on additional knowledge as in Table~\ref{tab:cond_ops}.

\begin{table}[h]
\begin{tabular}{|lll|}
\hline
Operators & SymPy function & example\\ \hline
probability & {\tt P} & {\tt P(X > 3)}  $\rightarrow 1/2$\\ \hline
expectation & {\tt E} & {\tt E(2*X)} $\rightarrow 7$ \\ \hline
variance & {\tt var} & {\tt var(X + 3)} $\rightarrow 35/12$\\ \hline
density & {\tt var} & {\tt Density(2*X)}  $\rightarrow \{2: 1/6,\; 4: 1/6,\; 6: 1/6,$ \\
            & & \phantom{Density(X+3) $\rightarrow$} $\quad 8: 1/6,\; 10: 1/6,\; 12: 1/6\}$ \\ \hline
\end{tabular}
\label{tab:stat_ops}
\caption{Operators over expressions using random variables}
\end{table}

\begin{table}[h]
\begin{tabular}{|lll|}
\hline
Explanation & SymPy function & example\\ \hline
Conditional random variable & {\tt Given} & {\tt X2 = Given(X, X < 3)} \\ \hline
Conditional density & {\tt Density} & {\tt Density(X, X > 3) } $\rightarrow \{4: 1/3,\; 5: 1/3,\; 6: 1/3\}$ \\ \hline
Conditional expectation  & {\tt E} & {\tt E(X, X > 3) } $\rightarrow 5$ \\ \hline
\end{tabular}
\label{tab:cond_ops}
\caption{Conditional operators over random expressions}
\end{table}

While we use the {\tt Die} function above as a syntactic sugar, one can
represent any probability space by providing the density function, a map
from value to probability summing to one. For example, we define a fair coin as follows:
\begin{lstlisting}
>>> density_map = {'heads':.5, 'tails':.5}
>>> coin = FiniteRV(density_map)
\end{lstlisting}

By introducing a random variable type into the SymPy algebraic system, we have
created a language for modeling and querying symbolic statistical systems.
While these simple examples using dice are minimal, they form the basis for more
sophisticated examples.  These simple operations allow us to model statistical
systems in a readable and intuitive manner.
