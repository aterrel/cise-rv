\section{Example 1 - Kinematics}

Consider shooting a cannon ball off of a cliff with a certain muzzle velocity, angle, height, etc.... We ask where will the ball land below?

We model this mathematical problem in SymPy

\begin{lstlisting}
    # Symbols for position, velocity, angle, time, and gravity
    >>> x0, y0, yf, v, theta, t, g = symbols('x0 y0 y_f v theta t g') 
    # x and y positions as a function of time
    >>> x = x0 + v*cos(theta)*t
    >>> y = y0 + v*sin(theta)*t + g/2*t**2
    # Solve y = yf for time to obtain the duration of the flight
    >>> t_impact = solve(y-yf, t)[1]
    # Final x value of cannon ball on impact
    >>> xf = x0 + v*cos(theta)*t_impact
\end{lstlisting}
This code yields the following solution

\begin{equation}
\label{eqn:kinematics_symbolic}
x_f = x_{0} - \frac{v \left(v \sin{\left (\theta \right )} + \sqrt{- 2 g y_{0} + 2 g y_f + v^{2} \sin^{2}{\left (\theta \right )}}\right) \cos{\left (\theta \right )}}{g}
\end{equation}

To obtain numeric results for a particular set of values we can replace the symbolic variables, $x_0, y_0, y_f, v, \theta, g$ with numeric ones

\begin{lstlisting}
>>> g = -9.8
>>> v, theta = 10, pi/4
>>> x0, y0, yf = 0, 10, 0
\end{lstlisting}

\begin{equation}
\label{eqn:kinematics_numeric}
x_f = \frac{250}{49} + \frac{50}{49} \sqrt{123} = 16.4189148024586
\end{equation}

The expression in Eqn. \ref{eqn:kinematics_symbolic} serves two purposes. Because it is a full description of the problem it can provide insight as to the relation of the variables to the result. It also serves as a staging ground to produce the numeric solution in Eqn. \ref{eqn:kinematics_numeric}. Once the mathematical model has been described no further conceptual work is necessary on the part of the programmer to produce numeric results. 

\subsection{statistics}

We now consider the statistical problem. Imagine that we are uncertain about the initial height of the cannon and that rather than knowing a precise value $y_0 = 10$ we instead had a distribution $y_0 \sim \N{10}{1}$. Due to this uncertainty we can no longer ask for the precise value of $x_f$ but may instead choose to ask for the expectation value, variance, or even the full distribution. 

Traditionally this is done either through painstaking algebra (for simple cases) or by building external codes to sample using monte carlo methods. We now show how SymPy can be used to model this statistical problem and provide solutions. 

Rather than defining $y_0$ as a symbolic or numeric variable, we will declare it to be a random variable distributed normally with mean 10 and standard deviation 1. 
\begin{lstlisting}
>>> y0 = Normal(10,1)
\end{lstlisting}

This variable acts in all ways like a SymPy symbol and so it is still possible to form expressions like Eqn. \ref{eqn:kinematics_symbolic}. Random variables become special when statistical functions, such as the expectation operator, are used.

\begin{lstlisting}
>>> E(xf)
\end{lstlisting}

This follows standard theory to produce the following integral
$$\int_{-\infty}^{\infty} \frac{25}{49} \frac{\left(\sqrt{\frac{98}{5} x_{1} + 50} + 5 \sqrt{2}\right) e^{- \frac{1}{2} \left(x_{1} -10\right)^{2}}}{\sqrt{\pi}}\, dx_{1}$$

This integral is challenging and the solution includes special functions. Fortunately algorithmic tools are capable of producing exact analytic solutions. 

$$\textrm{E}(x_f) = 16.4099...$$
$$\textrm{var}(x_f) = .2044...$$

In principle we could ask for the full analytically computed density of $x_f$ although the complexity prohibits any meaningful insight. 

\subsection{complex problems}

Imagine now a more difficult problem where more of the variables in our model were uncertain
\begin{lstlisting}
>>> v = Uniform(9,11.5)
>>> theta = Normal(pi/4, pi/20)
\end{lstlisting}

Our queries may also become more complex such as \textit{what is the probability that theta was greater than $\pi/4$ given that the ball landed past $x_f>16$?} , $P(\theta>\frac{\pi}{4} | x_f>16)$. Asking this question in SymPy is just as easy as the example above. 
\begin{lstlisting}
>>> P(theta>pi/4  ,  xf>16)
\end{lstlisting}
Solving the resultant integral however is substantially more challenging.

For situations like these exact solutions are often infeasible and monte carlo codes are built around the problem. Because this problem statement is now cleanly separated from the solution code it was trivial to connect a simple sampling engine onto SymPy stats allowing all such queries to be approximated using a simple syntax
\begin{lstlisting}
>>> P(theta>pi/4,  xf>16, samples=10000)
0.3428
\end{lstlisting}

