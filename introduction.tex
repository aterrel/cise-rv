\section{Introduction}

Real world simulations require not only the physical models that currently
dominate scientific computing but also the encapsulation of uncertainty, that
is the use of statistics.  Unfortunately, the difficulty of implementing the
physical model often takes most of the development efforts and adding
uncertainty to the models is yet another axis of requirements.

To address this challenge, we present the use of random variables at the
symbolic level with several examples of their use.  Just as many domain
specific languages are able to use symbolic methods to represent problems and
generate implementations, we are able to add statistics to a physical model by
augmenting the language such that the variables involved represent the
uncertainty.

Below we give three examples of increasing complexity that use the random
variable capabilities in SymPy, a popular symbolics package for the Python
programming language, see a summary by Joyner et al\cite{Joyner2011}.  The
first dice example gives an example to reacquaint the reader with some basic
statistics.  Second we have an example of modeling temperatures and
assimilating the physical model with data measurements. Finally, we give a
larger more complete example using a classic kinematics problem.
